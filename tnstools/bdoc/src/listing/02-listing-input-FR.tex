\documentclass[12pt,a4paper]{article}

\usepackage[utf8]{inputenc}
\usepackage[french]{babel, varioref}

% USING OTHER TOOLS %

\usepackage[lang = FR]{../main/main}
\usepackage{../macroenv/macroenv}
\usepackage{../inenglish/inenglish}
\usepackage{../showcase/showcase}
\usepackage{../rem-exa/rem-exa}

% TESTING LOCAL IMPLEMENTATION %

\usepackage{listing}


% == EXTRAS == %

% Source.
%    * https://tex.stackexchange.com/a/604698/6880

\NewDocumentCommand{ \bdocextraruler }{ m }{%
    \par
    {
        \centering
        \color{green!50!black}%
        \leavevmode
        \kern.075\linewidth
        \leaders\hrule height3.25pt\hfill\kern0pt
        \footnotesize\itshape\bfseries\space\ignorespaces#1\unskip\space
        \leaders\hrule height3.25pt\hfill\kern0pt
        \kern.075\linewidth
    	\par
	}
}

\NewDocumentEnvironment{ bdoc-extra-showcase }
                       { O{ Début du rendu dans cette doc. }
                         O{ Fin du rendu dans cette doc. } }{
    \begin{colorstrip}{green!5}
        \bdocextraruler{#1}
        \smallskip
}{
        \smallskip
        \bdocextraruler{#2}
    \end{colorstrip}
}


\begin{document}

%\section{Cas d'utilisation en \LaTeX}

\subsection{Codes importés}

Pour les codes suivants, on considère un fichier \verb+examples/listing-xyz.tex+ dont le chemin est donné relativement au document présent.
Le contenu de ce fichier est l'unique ligne \docilatex|$x y z = 1$| .

\medskip

Notez que les 1\ieres{} macros sont nommées de façon similaire aux environnements précédents en ajoutant le préfixe \docpre{input} tout en ignorant les tirets.


\begin{docrem}
	Il faut savoir que les macros imprimant automatiquement du texte tiennent code de la langue choisie lors du chargement du package \docpack{bdoc}.
\end{docrem}


% ------------------ %


\docexa[Face à face]

\begin{doclatex-alone}
\inputdoclatex{examples/listing-xyz.tex}
\end{doclatex-alone}

Ceci produit la mise en forme suivante.

\inputdoclatex{examples/listing-xyz.tex}


% ------------------ %


\docexa[À la suite]

\docmacro{inputdoclatexflat} produit un code à plat comme ci-dessous.

\inputdoclatexflat{examples/listing-xyz.tex}


% ------------------ %


\docexa[Juste le code]


\docmacro{inputdoclatexalone} sert à n'avoir que le code comme ci-après.

\inputdoclatexalone{examples/listing-xyz.tex}

% ------------------ %


%\newpage
\docexa[Code suivi du rendu centré]

Le rendu suivant est obtenu en utilisant \docmacro{inputdoclatexbefore}.

\medskip

\begin{docshowcase}
    \inputdoclatexbefore{examples/listing-xyz.tex}
\end{docshowcase}

\bigskip

Il est possible de changer le texte entre le code et son rendu via un argument optionnel.
Ainsi \docilatex|\inputdoclatexbefore[Voici ce que cela donne.]{...}| aboutit au résultat suivant.

\medskip

\begin{docshowcase}[Début du rendu réel \docquote{personnalisé}]%
                   [Fin du rendu réel \docquote{personnalisé}]
    \inputdoclatexbefore[Voici ce que cela donne.]{examples/listing-xyz.tex}
\end{docshowcase}


% ------------------ %


\docexa[Rendu centré suivi du code] \label{output-centered}

Le rendu suivant
\footnote{
	Il faut savoir que le 1\ier{} espace vertical disgracieux vient de l'emploi de \docenv{center} en coulisse.
},
similaire au précédent, est obtenu en appellant \docmacro{inputdoclatexafter} au lieu de \docmacro{inputdoclatexbefore}.

\medskip

\begin{docshowcase}
    \inputdoclatexafter{examples/listing-xyz.tex}
\end{docshowcase}

\bigskip

Via \docilatex|\inputdoclatexafter[Cette formule se tape comme suit.]{...}| , on obtient le résultat ci-après.

\medskip

\begin{docshowcase}[Début du rendu réel \docquote{personnalisé}]%
                   [Fin du rendu réel \docquote{personnalisé}]
    \inputdoclatexafter[Cette formule se tape comme suit.]{examples/listing-xyz.tex}
\end{docshowcase}


% ------------------ %


\docexa[Code importé et son rendu réel]

Pour un code et son rendu réel non centré, on utilisera \docmacro{inputdoclatexreal} qui va produire ce qui suit

\medskip

\begin{bdoc-extra-showcase}
    \inputdoclatexreal{examples/listing-xyz.tex}
\end{bdoc-extra-showcase}

\bigskip

Là aussi le texte par défaut peut être \docquote{personnalisé}. Par exemple, ce qui suit a été obtenu via \docilatex|\inputdoclatexreal[On obtient le rendu réel ci-après.]{...}| .

\medskip

\begin{bdoc-extra-showcase}[Début du rendu \docquote{personnalisé} dans cette doc.]%
                           [Fin du rendu \docquote{personnalisé} dans cette doc.]
    \inputdoclatexreal[On obtient le rendu réel ci-après.]{examples/listing-xyz.tex}
\end{bdoc-extra-showcase}

\end{document}
