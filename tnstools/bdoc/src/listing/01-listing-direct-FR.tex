\documentclass[12pt,a4paper]{article}

\usepackage[utf8]{inputenc}
\usepackage[french]{babel, varioref}

% USING OTHER TOOLS %

\usepackage[lang = FR]{../main/main}
\usepackage{../macroenv/macroenv}
\usepackage{../inenglish/inenglish}
\usepackage{../showcase/showcase}
\usepackage{../rem-exa/rem-exa}

% TESTING LOCAL IMPLEMENTATION %

\usepackage{listing}


% == EXTRAS == %

% Source.
%    * https://tex.stackexchange.com/a/604698/6880

\NewDocumentCommand{ \bdocextraruler }{ m }{%
    \par
    {
        \centering
        \color{green!50!black}%
        \leavevmode
        \kern.075\linewidth
        \leaders\hrule height3.25pt\hfill\kern0pt
        \footnotesize\itshape\bfseries\space\ignorespaces#1\unskip\space
        \leaders\hrule height3.25pt\hfill\kern0pt
        \kern.075\linewidth
    	\par
	}
}

\NewDocumentEnvironment{ bdoc-extra-showcase }
                       { O{ Début du rendu dans cette doc. }
                         O{ Fin du rendu dans cette doc. } }{
    \begin{colorstrip}{green!5}
        \bdocextraruler{#1}
        \smallskip
}{
        \smallskip
        \bdocextraruler{#2}
    \end{colorstrip}
}


\begin{document}

\section{Cas d'utilisation en \LaTeX}

\subsection{Codes \docquote{en ligne}}

La macro \docmacro{docilatex} permet de taper du code en ligne via un usage similaire à \docmacro{verb} comme dans les deux exemples suivants.

\begin{enumerate}
	\item \docilatex+\docilatex|$a^b = c$|+ produit \docilatex|$a^b = c$| .

	\item \docilatex+\docilatex|$a^b = c$|+ vient de \docilatex#\docilatex+\docilatex|$a^b = c$|+# .
\end{enumerate}


\begin{docrem}
	Le nom de la macro \docmacro{docilatex} vient de \docquote{\docprewhy{i}{nline} \LaTeX} soit \docinEN{\LaTeX\ en ligne}.
\end{docrem}


% ------------------ %


\subsection{Codes tapés directement}


% ------------------ %


\docexa[Face à face]

\begin{doclatex-alone}
\begin{doclatex}
    $A = B + C$
\end{doclatex}
\end{doclatex-alone}

Ceci donne :

\begin{doclatex}
    $A = B + C$
\end{doclatex}


% ------------------ %


\docexa[À la suite]

Via \docenv{doclatex-flat}, on obtient un code à plat
\footnote{
    Le suffixe \docpre{flat} signifie \docinEN{plat}.
}
comme ci-dessous.

\begin{doclatex-flat}
    $A = B + C$
\end{doclatex-flat}


% ------------------ %


\docexa[Juste le code]

\docenv{doclatex-alone} affiche le code seul
\footnote{
    Le suffixe \docpre{alone} signifie \docinEN{seul}.
}
comme ci-après.

\begin{doclatex-alone}
    $A = B + C$
\end{doclatex-alone}

\end{document}
