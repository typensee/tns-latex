\documentclass[12pt,a4paper]{article}

\usepackage[utf8]{inputenc}
\usepackage[french]{babel, varioref}

% USING OTHER TOOLS %

\usepackage[lang = FR]{../main/main}
\usepackage{../macroenv/macroenv}
\usepackage{../inenglish/inenglish}
\usepackage{../showcase/showcase}
\usepackage{../listing/listing}
\usepackage{../rem-exa/rem-exa}

% TESTING LOCAL IMPLEMENTATION %

\usepackage{version-n-change}


\begin{document}

\section{Changements}

\docexa[Expliquer]

\inputdoclatexreal{examples/logchange-topic.tex}


\begin{docrem}
    La valeur par défaut de l'argument optionnel de \docenv{doctopic} est \texttt{i} pour \docprewhy{i}{nline} soit \docinEN{en ligne}.
\end{docrem}


% ------------------ %


\docexa[Nouveautés]

Dans la documentation il est utile d'indiquer la date d'une nouveauté récente. Ceci se fait facilement via la macro \docmacro{docnew} comme dans l'exemple suivant où la date doit être donnée au format anglais \texttt{AAAA-MM-JJ}.

\inputdoclatexreal{examples/logchange-new.tex}


\begin{docrem}
	Deux compilations au minimum sont nécessaires.
\end{docrem}


% ------------------ %


\docexa[Numéro et date d'une version]

L'exemple suivant montre une macro plus complète que la précédente mais avec un fonctionnement similaire
\footnote{
	La macro \docmacro{docnew} a été conçue pour la documentation elle-même et \docmacro{docversion} juste pour l'historique des changements.
}.

\inputdoclatexreal{examples/logchange-version.tex}


\begin{docrem}
	\leavevmode

	\begin{enumerate}
		\item Le numéro de version sera du type \texttt{maj.min.bug} ou \texttt{maj.min.bug-extra} avec trois entiers \texttt{maj}, \texttt{min}, \texttt{bug} et un optionnel \texttt{extra} égal à \texttt{alpha}, \texttt{beta} ou \texttt{rc} pour \docquote{release candidate}.

		\item Noter dans le rendu final le format français \texttt{JJ/MM/AAAA} de la date alors que dans le code celle-ci doit être donnée au format anglais \texttt{AAAA-MM-JJ}.
	\end{enumerate}
\end{docrem}

\end{document}
