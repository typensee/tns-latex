\documentclass[12pt,a4paper]{article}

\usepackage[utf8]{inputenc}
\usepackage[french]{babel, varioref}

% USING OTHER TOOLS %

\usepackage[lang = FR]{../main/main}
\usepackage{../listing/listing}
\usepackage{../rem-exa/rem-exa}

% TESTING LOCAL IMPLEMENTATION % 

\usepackage{macroenv}


\begin{document}

\section{Indiquer des packages, des macros ou des environnements}

Voici ce qu'il est possible de taper de façon sémantique.

\begin{doclatex}
\docpack{monpackage} est pour...

\docmacro{unemacro} permet de...

\docenv{cetenv} sert à...
\end{doclatex}


\begin{docrem}
	L'intérêt des macros précédentes vis à vis de l'usage de \docmacro{docilatex} est l'absence de coloration.
	De plus, la macro \docmacro{docenv} demande juste de taper le nom de l'environnement
	\footnote{
		De plus, \docenv{cetenv} utilise des espaces pour autoriser des retours à la ligne si besoin.
	}.
\end{docrem}


% -------------------- %


\section{Expliquer les préfixes ou les suffixes}

Pour expliquer les noms retenus rien de tel que d'indiquer et expliciter les courts préfixes et suffixes retenus.

\begin{doclatex}
\docpre{pré} est...

\docprewhy{suf}{fixe} explique...
\end{doclatex}



\end{document}

