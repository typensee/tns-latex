\documentclass[12pt, a4paper]{article}

\newcommand\thispack{\tdocpack{tns-functab}}


\usepackage[utf8]{inputenc}
\usepackage[T1]{fontenc}

\usepackage{multicol}
\usepackage[inline]{enumitem}

\usepackage[french]{babel, varioref}
\frenchsetup{StandardItemLabels=true}

\usepackage[lang = french]{tutodoc}

\usepackage{amsmath}
\usepackage[locale=FR]{siunitx}
\usepackage{tutodoc}


\usepackage[lang = french]{../main/main}
\usepackage{../macroenv/macroenv}
\usepackage{../inenglish/inenglish}
\usepackage{../focus/focus}


\begin{document}

Le package \bdocpack{bdoc}
\footnote{
    Le nom vient de \bdocquote{\bdocprewhy{b.asic} \bdocprewhy{doc.umentation}} qui ne nécessite aucune traduction.
}
est sans aucune prétention.
Son but est de faciliter la saisie sémantique de documentations de packages et de classes \LaTeX\ avec un rendu sobre pour une lecture sur écran
\footnote{
    L'idée est de produire un fichier \texttt{PDF} efficace à parcourir pour des besoins ponctuels. C'est généralement ce que l'on attend d'une documentation liée au codage.
}.


\begin{bdocnote}
 	Ce package propose, ou impose, un style de mise en forme.
	Dans un avenir plus ou moins proche, \bdocpack{bdoc} sera sûrement éclaté en une classe et un package.
\end{bdocnote}

\end{document}
