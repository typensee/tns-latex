\documentclass[10pt, a4paper]{article}

\usepackage[utf8]{inputenc}
\usepackage[T1]{fontenc}

\usepackage[french]{babel, varioref}

\usepackage{enumitem}
\frenchsetup{StandardItemLabels=true}

% Package documented.
\usepackage[lang = french]{tns-functab}



\begin{document}

\title{Le package \texttt{tutodoc} - Documentation de type tutoriel}
\author{Christophe BAL}
\date{1\ier{} Janv. 2024 - Version 1.1.0}

\maketitle

\begin{abstract}
Le package \tdocpack{tutodoc}
\footnote{
    Le nom vient de \tdocquote{\tdocprewhy{tuto.rial-type} \tdocprewhy{doc.umentation}} se traduit en \tdocquote{documentation de type tutoriel}.
}
est utilisé par son auteur pour produire de façon sémantique des documentations de packages et de classes \LaTeX\ dans un style de type tutoriel
\footnote{
    L'idée est de produire un fichier \texttt{PDF} efficace à parcourir pour des besoins ponctuels. C'est généralement ce que l'on attend d'une documentation liée au codage.
},
et avec un rendu sobre pour une lecture sur écran.


\begin{tdocnote}
     Ce package impose un style de mise en forme.
    Dans un avenir plus ou moins proche, \tdocpack{tutodoc} sera sûrement éclaté en une classe et un package.
\end{tdocnote}

\tdocsep

{\small\itshape
\textbf{Abstract.}
The \tdocpack{tutodoc} package
\footnote{
    The name comes from \tdocquote{\tdocprewhy{tuto.rial-type} \tdocprewhy{doc.umentation}}.
}
is used by its author to semantically produce documentation of \LaTeX\ packages and classes in a tutorial style
\footnote{
    The idea is to produce an efficient \texttt{PDF} file that can be browsed for one-off needs. This is generally what is expected of coding documentation.
},
and with a sober rendering for reading on screen.


\begin{tdocnote}
     This package imposes a formatting style. In the not-too-distant future, \tdocpack{tutodoc} will probably be split into a class and a package.
\end{tdocnote}
}
\end{abstract}


\newpage
\tableofcontents
\newpage
\section{Des tableaux de données}

Nous allons voir dans cette section comment saisir des tableaux de données de l'un des types suivants.
\begin{itemize}
	\item Tableaux d'images de plusieurs fonctions à une variable.

	\item Tableaux d'images d'une seule fonction à deux variables.
\end{itemize}


\begin{tdocnote}
	Tous les tableaux seront fabriqués via \tdocenv{functab} en utilisant un langage spécifique simplifiant la saisie des informations.
	Cet environnement est assez \tdocquote{malin} pour deviner le type de tableau souhaité en fonction des instructions fournies comme nous le constaterons dans les sections qui vont suivre.
\end{tdocnote}



% --------------- %


\subsection{Tableaux d'images de plusieurs fonctions à une variable}

Pour les tableaux de données, il faut commencer par indiquer les valeurs \tdocquote{initiales}
\footnote{
	 On peut indiquer soit des valeurs dont on veut donner les images, soit des bornes d'intervalles pour les signes et/ou les variations d'une fonction réelle.
} :
dans le cas des images de fonctions à une variable, cela se fait via
\tdocinlatex{xvals = mavar : x_1 , x_2 , ... , x_n}
ou juste
\tdocinlatex{xvals = x_1 , x_2 , ... , x_n}
si la variable par défaut, à savoir $x$, convient.
Une fois ceci fait, il faut renseigner les différentes images via \tdocinlatex{imgs = monexpr : im_1 , im_2 , ... , im_n} en donnant obligatoirement la formule de l'expression étudiée.
Voici des cas d'utilisation.


% --------------- %


\begin{tdocexa}[Une seule fonction avec la variable par défaut]
    \leavevmode

    \tdoclatexinput{examples-1D-data-one-func-no-label.tex}
\end{tdocexa}


\begin{tdocnote}
	Retenir que tout se saisie en mode mathématique.
\end{tdocnote}


\begin{tdocwarn}
	L'utilisation de \tdocinlatex{xvals} doit se faire obligatoirement une fois, et une seule, au tout début du contenu.
\end{tdocwarn}


% --------------- %


\begin{tdocexa}[Deux fonctions pour une variable \tdocquote{maison}]
    \leavevmode

    \tdoclatexinput{examples-1D-data-two-func-label.tex}
\end{tdocexa}


% --------------- %


\begin{tdocexa}[Commentaires à la sauce \LaTeX]
    \leavevmode

    \tdoclatexinput{examples-1D-data-comments.tex}
\end{tdocexa}


% --------------- %

\begin{tdoctip}[Nombres décimaux en version \tdocquote{locale} et \tdocquote{grandes} fractions]
%    \leavevmode
%
    Via les macros \tdocmacro{dfrac} et \tdocmacro{num}
    \footnote{
    	Cette macro ajoute de fins espaces mettant en valeur les groupes de chiffres, tout en gérant l'absence d'espaces autour du séparateur décimal.
    }
    venant des excellents packages \tdocpack{amsmath} et \tdocpack{siuntix} respectivement, il est facile de rédiger des nombres décimaux, et d'obtenir de \tdocquote{grandes} fractions comme le montre l'exemple suivant.

    \tdoclatexinput{examples-1D-data-num-n-dfrac.tex}
\end{tdoctip}


%\section{Des tableaux de données}

\subsection{Tableaux d'images d'une seule fonction à deux variables}

Voyons comment taper un tableau d'images d'une fonction à deux variables, ou de façon équivalente d'un tableau à double-entrée
\footnote{
	Bien entendu, on perd ici la possibilité de travailler avec plusieurs fonctions au sein du même tableau.
}
toujours via l'environnement \tdocenv{functab} qui est multifonctions.


% --------------- %


\begin{tdocexa}[Avec des étiquettes de partout]
    En plus de \tdocinlatex{xvals} pour les colonnes, il faut utiliser \tdocinlatex{yvals}, avec un fonctionnement similaire, pour renseigner les valeurs initiales qui seront utilisées pour chaque ligne.
    Quand aux valeurs de chaque cellule, on les donne sous forme matricielle de façon naturelle via \tdocinlatex{mat} : pour une ligne, les valeurs sont séparées par des virgules, tandis que les points-virgules permettent de passer à la ligne suivante.

    \tdoclatexinput{examples-2D-data-label.tex}
\end{tdocexa}


% --------------- %


\begin{tdocexa}[Sans indiquer les étiquettes]
    \leavevmode

    \tdoclatexinput{examples-2D-data-no-label.tex}
\end{tdocexa}


\section{Tableaux de signes}

\begin{tdocnote}
	Tout comme pour les tableaux de données, c'est via \tdocenv{functab} et un langage spécifique que les tableaux de signes seront fabriqués.
\end{tdocnote}


% --------------- %


\subsection{Le cas des fonctions réelles d'une variable réelle}

La saisie d'un tableau de signes se veut être \tdocquote{naturelle} et \tdocquote{facile}\,. Les exemples suivants devraient vous convaincre que c'est bien le cas.


%\section{Tableaux de signes}

\subsection{Le cas des courbes planes paramétrées}

La saisie de tableaux de signes pour des courbes paramétrées nécessite d'afficher la ligne des valeurs pivots aussi en fin de tableau : ceci s'obtient via \tdocenv[{[param = true]}]{functab} : ceci demande . Voici un exemple d'utilisation de cette option.


\section{Tableaux de variations}

\begin{tdocnote}
	Pour les variations, c'est de nouveau via \tdocenv{functab} et un langage spécifique que les tableaux seront fabriqués.
\end{tdocnote}


% --------------- %


\subsection{Le cas des fonctions réelles d'une variable réelle}

La saisie d'un tableau de variations est moins immédiate que celle d'un tableau de signes puisque l'on peut indiquer plus d'informations. Les exemples ci-après ont pour mission de rendre \tdocquote{non douloureuse} la saisie de tableaux de variations à la sauce \thispack.


%\section{Tableaux de variations}

\subsection{Le cas des courbes planes paramétriques}

Tout comme pour les tableaux de signes, via l'option \tdocinlatex{param = true}, un tableau de variations contiendra dans sa dernière ligne les valeurs pivots. Voici un exemple d'utilisation de cette option.


\section{Tableaux de signes et des variations}

Afin d'étudier des fonctions réelles via les signes de leur dérivée, il est possible de mixer des lignes de signes et d'autres pour les variations. Voici des cas d'utilisation.


\section{Syntaxe du langage interprétable par \LaTeX}
\label{tns-math-functab-dsl-l3}

Comme ceci a été montré dans les sections précédentes, \thispack\ facilite la saisie de différents types de tableaux liés à des fonctions en proposant un langage de saisie intuitif
\footnote{
    En informatique, on parle de \texttt{DSL} soit \tdocquote{Domain Specific Language}\,, c'est-à-dire \tdocinEN{Langue spécifique à un domaine}\,.
}.
Cette section regroupe les règles syntaxiques et sémantiques de ce langage.


%\section{La syntaxe du langage proposé}

\subsection{Les blocs d'instruction}

L'organisation des blocs d'informations se fait en respectant les règles suivantes.
%
\begin{enumerate}
    \item Les commentaires se font uniquement à la sauce \LaTeX\ via \tdocinlatex{% Mon commentaire} par exemple.


    \item Les lignes vides sont autorisées
    \footnote{
        \label{tns-math-functab-dsl-l3-blank-struct}%
        Ceci est très pratique pour aérer et organiser le code.
    }.


    \item La syntaxe pour les contenus est la suivante.
    %
    \begin{itemize}
        \item Les différents contenus sont séparés par des points-virgules, les retours à la ligne étant autorisés.

        \item Les espaces initiaux et finaux sont ignorés (voir la note \ref{tns-math-functab-dsl-l3-blank-struct} de bas de page).

        \item Tout contenu est associé à un contexte.

        \item Un nouveau contexte s'indique en début de contenu via \tdocinlatex{nom_contexte = contenu_1 ; ...} sans nécessité d'indiquer à chaque fois le contexte.

        \item \textbf{Aucun contexte par défaut n'est initialement défini.}
    \end{itemize}


    \item Le point-virgule est optionnel à la fin du tout dernier contenu.
\end{enumerate}


% --------------- %


\medskip


Voici un code fictif illustrant les règles précédentes.

\tdoclatexinput[code]{examples-dsl-l3-struct-fictional.tex}


Il est possible d'écrire des contenus à la suite comme dans l'exemple suivant
\footnote{
    Nous déconseillons ce style de codage car cela nuit à la lisibilité des instructions.
}.

\tdoclatexinput[code]{examples-dsl-l3-struct-fictional-compact.tex}


On peut même utiliser l'horreur suivante
\footnote{
    Ce mode de rédaction est en fait idéal pour la fabrication automatisée de codes par un script extérieur.
}.

\tdoclatexinput[code]{examples-dsl-l3-struct-fictional-monster.tex}


%\section{La syntaxe du langage proposé}

\subsection{Les contextes pour les images de plusieurs fonctions à une variable}

Voici les contextes utilisables pour les images de plusieurs fonctions à une variable.

\medskip

\begin{itemize*}[before = \leavevmode\kern15pt, itemjoin = \kern15pt]
	\item \tdocinlatex{xvals}

	\item \tdocinlatex{imgs}
\end{itemize*}


%\section{La syntaxe du langage proposé}
%\subsection{Les contextes pour les images de plusieurs fonctions à une variable}

\subsubsection{Le contexte \tdocinlatex{xvals} pour les valeurs étudiées}
\label{tns-math-functab-dsl-l3-ctxt-xvals}

Pour indiquer les valeurs nécessaires à la conception d'un tableau, il faut passer via le contexte \tdocinlatex{xvals} en respectant les règles ci-après.
%
\begin{enumerate}
    \item Le contexte \tdocinlatex{xvals} doit être employé une seule fois en tout début de code, hors commentaires, c'est-à-dire avant le contexte \tdocinlatex{imgs}.


    \item Un seul contenu est utilisable en utilisant l'un des deux formats suivants où le nombre de valeurs n'est pas majoré.
    %
    \begin{itemize}
        \item \tdocinlatex{ma_{var} : x_1 , x_2 , ... , x_n} indique que les $n$ valeurs \tdocinlatex{x_1} , \tdocinlatex{x_2} , \dots\ , \tdocinlatex{x_n} ont été données pour la variable nommée $ma_{var}$ avec $n \geq 2$ forcément.

        \item \tdocinlatex{x_1 , x_2 , ... , x_n} indique que $n$ valeurs pivots \tdocinlatex{x_1} , \tdocinlatex{x_2} , \dots\ , \tdocinlatex{x_n} ont été données pour la variable nommée $x$ qui est le nom par défaut.
    \end{itemize}
\end{enumerate}


\medskip


Voici deux codes fictifs illustrant les règles précédentes.
\begin{multicols}{2}
    \tdoclatexinput[code]{examples-dsl-l3-xvals-label.tex}

    \tdoclatexinput[code]{examples-dsl-l3-xvals-no-label.tex}
\end{multicols}


% --------------- %


\begin{tdocnote}
    Les espaces autour des doubles points et des virgules ne sont pas obligatoires bien que conseillés pour des raisons de lisibilité.
\end{tdocnote}


%\section{La syntaxe du langage proposé}
%\subsection{Les contextes pour les images de plusieurs fonctions à une variable}

\subsubsection{Le contexte \tdocinlatex{imgs} pour les images}

Le contexte \tdocinlatex{imgs} s'utilise de façon similaire au contexte \tdocinlatex{xvals} mais avec la possibilité de l'employer autant de fois que nécessaire.
Par contre, il est obligatoire d'indiquer l'expression  via
\tdocinlatex{mon_expr : img_1 ,}
\tdocinlatex{img_2 , ... , img_n}
où chaque \tdocinlatex{img_k} peut être soit \tdocquote{vide}\,, soit une expression mathématique \LaTeX\,.
\textbf{Bien retenir qu'il n'existe pas d'expression par défaut.}


% --------------- %


\begin{tdocnote}
    Les espaces autour des doubles points et des virgules ne sont pas obligatoires.
\end{tdocnote}


% --------------- %


Voici un code fictif illustrant les explications précédentes.

\tdoclatexinput[code]{examples-dsl-l3-imgs-f-n-co.tex}


Dans le cadre de processus automatisés, il est possible de produire l'horreur suivante qui aboutira à la même sortie que le code ci-dessus.

\tdoclatexinput[code]{examples-dsl-l3-imgs-f-n-co-monster.tex}


%\section{La syntaxe du langage proposé}

\subsection{Les contextes pour les signes et/ou les variations}

Voici les contextes utilisables pour les signes et/ou les variations.

\medskip

\begin{itemize*}[before = \leavevmode\kern15pt, itemjoin = \kern15pt]
	\item \tdocinlatex{xvals}

	\item \tdocinlatex{signs}

	\item \tdocinlatex{vars}
\end{itemize*}


% --------------- %


\subsubsection{Le contexte \tdocinlatex{xvals} pour les valeurs pivots}

Se reporter à la section \ref{tns-math-functab-dsl-l3-ctxt-xvals} donnant la syntaxe attendue ; ce qui échange ici est que les valeurs données sont des valeurs pivots liées aux intervalles sur lesquels on donne des informations.


%\section{La syntaxe du langage proposé}
%\subsection{Les contextes pour les signes et/ou les variations}

\subsubsection{Le contexte \tdocinlatex{signs} pour les signes}

Voici tous les éléments de syntaxe propres au contexte \tdocinlatex{signs}.
%
\begin{enumerate}
    \item Le contexte \tdocinlatex{signs} est employable autant de fois que nécessaire.


    \item L'expression dont on indique le signe doit être donnée via \tdocinlatex{mon_expr : ...} où les points de suspension spécifient les informations liées au comportement de l'expression (voir l'item suivant à ce sujet).
    \textbf{Il n'existe pas d'expression par défaut.}


    \item Dans cet item, pour expliquer ce qui est attendu comme informations relatives au comportement d'une expression $f$ , nous allons supposer que $n$ valeurs pivots \tdocinlatex{x_1} , \tdocinlatex{x_2} , \dots\ , \tdocinlatex{x_n} ont été données via \tdocinlatex{xvals = x_1 , x_2 , ... , x_n} . Nous posons aussi $p = n - 1$ avec \tdocpre{p} pour \tdocquote{\tdocprewhy{p.récédent} le naturel $n$} .
    %
    \begin{itemize}
        \item A minima, il faut indiquer
        \tdocinlatex{f : s_1 s_2 ... s_p}
        où
        \tdocinlatex{s_1} , \tdocinlatex{s_2} , \dots\ , \tdocinlatex{s_p}
        donnent des informations sur les $p$ intervalles
        \tdocinlatex{]x_1 ; x_2[} , \tdocinlatex{]x_2 ; x_3[} , \dots\ , \tdocinlatex{]x_p ; x_n[} respectivement.
        Les valeurs possibles pour les \tdocinlatex{s_k} sont les suivantes.
        %
        \begin{enumerate}
            \item \tdocinlatex{+} indique une expression positive stricte sur l'intervalle concerné.

            \item \tdocinlatex{-} indique une expression négative stricte sur l'intervalle concerné.

            \item \tdocinlatex{z} indique une expression nulle sur l'intervalle concerné avec \tdocpre{z} pour \tdocprewhy{z.éro}.

            \item \tdocinlatex{u} indique une expression non définie sur l'intervalle concerné
            \footnote{
                Penser par exemple à l'expression $x \, \sqrt{x^2 - 1}$ .
            }
            avec \tdocpre{u} pour \tdocprewhy{u.ndefined} soit \tdocinEN{non défini}.
        \end{enumerate}


        \item On peut aussi indiquer le comportement d'une expression en certaines valeurs pivots. Ceci se fait à côté d'une information de type signe : par exemple, en gardant les notations de l'item précédent, nous avons les possibilités suivantes.
        %
        \begin{itemize}
            \item Si $n > 3$ alors $f(x_3) = 0$ s'indique via \tdocinlatex{... 0 s_3 ...} au milieu du code.

            \item Si $n = 3$ alors $f(x_3) = 0$ s'indique via \tdocinlatex{... s_2 0} en fin de code.

            \item $f(x_1) = 0$ s'indique via \tdocinlatex{0 s_1 ...}    au début du code.
        \end{itemize}
        %
        Les valeurs possibles pour le comportement éventuel en un pivot sont les suivantes.
        %
        \begin{enumerate}
            \item \tdocinlatex{0} indique que
l'expression s'annule au pivot concerné.

            \item \tdocinlatex{!} indique que
l'expression n'est pas définie au pivot concerné
            \footnote{
                En France, le panneau de signalisation indiquant un danger est un triangle blanc au bords rouges rouge et contenant un point d'exclamation.
            }.

            \item L'absence de valeur indique que l'expression ne change pas de signe au pivot concerné.
        \end{enumerate}
    \end{itemize}
\end{enumerate}


% --------------- %


%Voici un tableau permettant de mémoriser l'organisation des informations de type \tdocquote{signe} et celles de type \tdocquote{pivot}.


% --------------- %


\begin{tdocnote}
    Les espaces autour des doubles points et des informations codées ne sont pas obligatoires.
\end{tdocnote}


% --------------- %


Voici deux codes fictifs illustrant les explications précédentes ; noter au passage que les espaces ignorés permettent d'obtenir un résultat humainement très clair.
\begin{multicols}{2}
    \tdoclatexinput[code]{examples-dsl-l3-signs-plus-minus-zero.tex}

    \tdoclatexinput[code]{examples-dsl-l3-signs-plus-minus-strange.tex}
\end{multicols}

Dans le cadre de processus automatisés, il est possible de produire les horreurs suivantes qui aboutiront aux mêmes sorties que les codes correspondants ci-dessus.
\begin{multicols}{2}
    \tdoclatexinput[code]{examples-dsl-l3-signs-plus-minus-zero-monster.tex}

    \tdoclatexinput[code]{examples-dsl-l3-signs-plus-minus-strange-monster.tex}
\end{multicols}


%\section{La syntaxe du langage proposé}
%\subsection{Les contextes pour les signes et/ou les variations}

\subsubsection{Le contexte \tdocinlatex{vars} pour les variations}

La logique d'utilisation et d'organisation des informations pour le contexte \tdocinlatex{vars} est similaire à celle du contexte \tdocinlatex{signs} ; nous indiquons donc juste les différences.
%
\begin{enumerate}
    \item Voici les informations possibles pour le comportement d'une expression $f$ en supposant que $n$ valeurs pivots \tdocinlatex{x_1} , \tdocinlatex{x_2} , \dots\ , \tdocinlatex{x_n} ont été données via \tdocinlatex{xvals = x_1 , x_2 , ... , x_n} . De nouveau, nous posons $p = n - 1$ .
    %
    \begin{itemize}
        \item A minima, il faut indiquer
        \tdocinlatex{f : v_1 v_2 ... v_p}
        où
        \tdocinlatex{v_1} , \tdocinlatex{v_2} , \dots\ , \tdocinlatex{v_p}
        donnent des informations sur les $p$ intervalles ouverts
        \tdocinlatex{]x_1 ; x_2[} , \tdocinlatex{]x_2 ; x_3[} , \dots\ , \tdocinlatex{]x_p ; x_n[} respectivement.
        Les valeurs possibles pour les \tdocinlatex{v_k} sont les suivantes.
        %
        \begin{enumerate}
            \item \tdocinlatex{<} indique une expression strictement croissante sur l'intervalle concerné.

            \item \tdocinlatex{>} indique une expression strictement décroissante sur l'intervalle concerné.

            \item \tdocinlatex{=} indique une expression constante sur l'intervalle concerné.

            \item \tdocinlatex{u} indique une expression non définie sur l'intervalle concerné (comme pour les signes).
        \end{enumerate}


        \item Pour des valeurs pivots précises, on peut indiquer des images ou des limites à droite et/ou à gauche. Voici ce qui est disponible.
        %
        \begin{enumerate}
            \item L'absence d'expression est possible pour ne rien indiquer du tout.

            \item Toute expression sans lettre \tdocinlatex{u}, ni ponctuation \tdocinlatex{!} est interprétée comme une valeur image au format mathématique \LaTeX.

            \item Une valeur image contenant \tdocinlatex{u} et/ou la ponctuation \tdocinlatex{!} en tant que \tdocquote{token \LaTeX} devra être protégée par des accolades. Ceci vient du fait que dans le langage codant les variations ces deux caractères ont une signification spéciale (voir ci-dessus et l'item suivant).

            \item \tdocinlatex{!} indique que
l'expression n'est pas définie au pivot concerné.
            On peut aussi, si besoin, indiquer des limites à gauche et/ou à droite. Voici les cas possibles.
            %
            \begin{itemize}
                \item \tdocinlatex{!} utilisé seul n'indique aucune limite.


                \item  \tdocinlatex{! d} indique juste $d$ comme limite à droite.

                \textbf{Cette syntaxe est interdite pour le tout dernier pivot.}


                \item \tdocinlatex{g !} indique juste $g$ comme limite à gauche.

                \textbf{Cette syntaxe est interdite pour le tout premier pivot.}


                \item  \tdocinlatex{g ! d} indique $g$ et $d$ comme limites à gauche et à droite respectivement.

                \textbf{Cette syntaxe est interdite pour les premier et dernier pivots.}
            \end{itemize}
        \end{enumerate}
    \end{itemize}
\end{enumerate}


% --------------- %


%Voici un tableau permettant de mémoriser l'organisation des informations de type \tdocquote{variation} et celles de type \tdocquote{pivot}.


% --------------- %


\begin{tdocnote}
    Les espaces autour des doubles points, du point d'exclamation et des informations codées ne sont pas obligatoires.
\end{tdocnote}


% --------------- %


Voici deux codes fictifs illustrant les explications précédentes ; noter au passage que les espaces ignorés permettent d'obtenir un résultat humainement très clair.
\begin{multicols}{2}
    \tdoclatexinput[code]{examples-dsl-l3-vars-always-defined.tex}

    \tdoclatexinput[code]{examples-dsl-l3-vars-with-undefined.tex}
\end{multicols}

Dans le cadre de processus automatisés, il est possible de produire les horreurs suivantes qui aboutiront aux mêmes sorties que les codes correspondants ci-dessus.
\begin{multicols}{2}
    \tdoclatexinput[code]{examples-dsl-l3-vars-always-defined-monster.tex}

    \tdoclatexinput[code]{examples-dsl-l3-vars-with-undefined-monster.tex}
\end{multicols}


%\section{La syntaxe du langage proposé}

\subsection{Les contextes pour les images d'une seule fonction à deux variables}

Voici les contextes utilisables pour les images d'une seule fonction à deux variables.

\medskip

\begin{itemize*}[before = \leavevmode\kern15pt, itemjoin = \kern15pt]
	\item \tdocinlatex{xvals}

	\item \tdocinlatex{yvals}

	\item \tdocinlatex{mat}
\end{itemize*}


% --------------- %


\subsubsection{Le contexte \tdocinlatex{xvals} pour la 1ière variable}

Se reporter à la section \ref{tns-math-functab-dsl-l3-ctxt-xvals} donnant la syntaxe attendue, mais avec une petite différence : le contexte \tdocinlatex{xvals} peut être précédé du contexte \tdocinlatex{yvals} présenté dans la section suivante.


%\section{La syntaxe du langage proposé}
%\subsection{Les contextes pour les images d'une seule fonction à deux variables}

\subsubsection{Le contexte \tdocinlatex{yvals} pour la 2ième variable}

La logique est la même que pour le contexte \tdocinlatex{xvals}, de nouveau en ne pouvant placer que le contexte \tdocinlatex{xvals} avant le contexte \tdocinlatex{yvals}.


%\section{La syntaxe du langage proposé}
%\subsection{Les contextes pour les images d'une seule fonction à deux variables}

\subsubsection{Le contexte \tdocinlatex{mat} pour les images}

Le contexte \tdocinlatex{mat} qui n'est utilisable qu'une seule fois, permet de donner la matrice des données, et ceci en indiquant, si besoin uniquement
\footnote{
	Ceci permet de taper un simple tableau à double entrée.
},
l'expression étudiée.
Voici un code fictif illustrant la logique de saisie.

\tdoclatexinput[code]{examples-dsl-l3-mat.tex}


% --------------- %

Comme les espaces autour des ponctutations ne sont pas obligatoires, dans le cadre de processus automatisés, il sera possible de produire l'horreur suivante qui aboutira à la même sortie que le code ci-dessus.

\tdoclatexinput[code]{examples-dsl-l3-mat-monster.tex}
\section{Historique}

\tdocversion{1.1.0}[2024-01-06]

\begin{tdocnew}
	\item Journal des changements : deux nouveaux environnements.
    \begin{enumerate}
        \item \tdocenv{tdocbreak} pour les \tdocquote{bifurcations}\,, soit les modifications non rétrocompatibles.

        \item \tdocenv{tdocprob} pour les problèmes repérés.
    \end{enumerate}

	\item \tdocmacro{tdocinlatex}: un jaune léger est utilisé comme couleur de fond.
\end{tdocnew}

\tdocsep

\tdocversion{1.0.1}[2023-12-08]

\begin{tdocfix}
	\item \tdocmacro{tdocenv}: l'espacement est maintenant correct, même si le paquet \tdocpack{babel} n'est pas chargé avec la langue française.

	\item \tdocenv[{[nostripe]}]{tdocshowcase}: les sauts de page autour des lignes \tdocquote{cadrantes} devraient être rares dorénavant.
\end{tdocfix}

\tdocsep

\tdocversion{1.0.0}[2023-11-29]

Première version publique du projet.

\end{document}
