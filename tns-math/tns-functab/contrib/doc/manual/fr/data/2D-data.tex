\documentclass[12pt, a4paper]{article}

\usepackage[utf8]{inputenc}
\usepackage[T1]{fontenc}

\usepackage[french]{babel, varioref}

\usepackage{enumitem}
\frenchsetup{StandardItemLabels=true}

\usepackage[lang = french]{tutodoc}



\begin{document}


%\section{Des tableaux de données}

\subsection{Tableaux d'images d'une seule fonction à deux variables}

Voyons comment taper un tableau d'images d'une fonction à deux variables, ou de façon équivalente d'un tableau à double-entrée
\footnote{
	Bien entendu, on perd ici la possibilité de travailler avec plusieurs fonctions au sein du même tableau.
}
toujours via l'environnement \tdocenv{functab} qui est multifonctions.


% --------------- %


\begin{tdocexa}[Avec des étiquettes de partout]
    En plus de \tdocinlatex{xvals} pour les colonnes, il faut utiliser \tdocinlatex{yvals}, avec un fonctionnement similaire, pour renseigner les valeurs initiales qui seront utilisées pour chaque ligne.
    Quand aux valeurs de chaque cellule, on les donne sous forme matricielle de façon naturelle via \tdocinlatex{mat} : pour une ligne, les valeurs sont séparées par des virgules, tandis que les points-virgules permettent de passer à la ligne suivante.

    \tdoclatexinput{examples/2D-data/label.tex}
\end{tdocexa}


% --------------- %


\begin{tdocexa}[Sans indiquer les étiquettes]
    \leavevmode

    \tdoclatexinput{examples/2D-data/no-label.tex}
\end{tdocexa}


\end{document}


\end{document}
    