\documentclass[12pt, a4paper]{article}

\usepackage[utf8]{inputenc}
\usepackage[T1]{fontenc}

\usepackage[french]{babel, varioref}

\usepackage{enumitem}
\frenchsetup{StandardItemLabels=true}

\usepackage[lang = french]{tutodoc}



\begin{document}


\section{Des tableaux de données}

Nous allons voir dans cette section comment saisir des tableaux de données de l'un des types suivants.
\begin{itemize}
	\item Tableaux d'images de plusieurs fonctions à une variable.

	\item Tableaux d'images d'une seule fonction à deux variables.
\end{itemize}


\begin{tdocnote}
	Tous les tableaux seront fabriqués via \tdocenv{functab} en utilisant un langage spécifique simplifiant la saisie des informations.
	Cet environnement est assez \tdocquote{malin} pour deviner le type de tableau souhaité en fonction des instructions fournies comme nous le constaterons dans les sections qui vont suivre.
\end{tdocnote}



% --------------- %


\subsection{Tableaux d'images de plusieurs fonctions à une variable}

Pour les tableaux de données, il faut commencer par indiquer les valeurs \tdocquote{initiales}
\footnote{
	 On peut indiquer soit des valeurs dont on veut donner les images, soit des bornes d'intervalles pour les signes et/ou les variations d'une fonction réelle.
} :
dans le cas des images de fonctions à une variable, cela se fait via
\tdocinlatex{xvals = mavar : x_1 , x_2 , ... , x_n}
ou juste
\tdocinlatex{xvals = x_1 , x_2 , ... , x_n}
si la variable par défaut, à savoir $x$, convient.
Une fois ceci fait, il faut renseigner les différentes images via \tdocinlatex{imgs = monexpr : im_1 , im_2 , ... , im_n} en donnant obligatoirement la formule de l'expression étudiée.
Voici des cas d'utilisation.


% --------------- %


\begin{tdocexa}[Une seule fonction avec la variable par défaut]
    \leavevmode

    \tdoclatexinput{examples/1D-data/one-func-no-label.tex}
\end{tdocexa}


\begin{tdocnote}
	Retenir que tout se saisie en mode mathématique.
\end{tdocnote}


\begin{tdocwarn}
	L'utilisation de \tdocinlatex{xvals} doit se faire obligatoirement une fois, et une seule, au tout début du contenu.
\end{tdocwarn}


% --------------- %


\begin{tdocexa}[Deux fonctions pour une variable \tdocquote{maison}]
    \leavevmode

    \tdoclatexinput{examples/1D-data/two-func-label.tex}
\end{tdocexa}


% --------------- %


\begin{tdocexa}[Commentaires à la sauce \LaTeX]
    \leavevmode

    \tdoclatexinput{examples/1D-data/comments.tex}
\end{tdocexa}


% --------------- %

\begin{tdoctip}[Nombres décimaux en version \tdocquote{locale} et \tdocquote{grandes} fractions]
%    \leavevmode
%
    Via les macros \tdocmacro{dfrac} et \tdocmacro{num}
    \footnote{
    	Cette macro ajoute de fins espaces mettant en valeur les groupes de chiffres, tout en gérant l'absence d'espaces autour du séparateur décimal.
    }
    venant des excellents packages \tdocpack{amsmath} et \tdocpack{siuntix} respectivement, il est facile de rédiger des nombres décimaux, et d'obtenir de \tdocquote{grandes} fractions comme le montre l'exemple suivant.

    \tdoclatexinput{examples/1D-data/num-n-dfrac.tex}
\end{tdoctip}

\end{document}


\end{document}
    