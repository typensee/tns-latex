\documentclass[12pt, a4paper]{article}

\newcommand\thispack{\tdocpack{tns-functab}}


\usepackage[utf8]{inputenc}
\usepackage[T1]{fontenc}

\usepackage{multicol}
\usepackage[inline]{enumitem}

\usepackage[french]{babel, varioref}
\frenchsetup{StandardItemLabels=true}

\usepackage[lang = french]{tutodoc}

\usepackage{amsmath}
\usepackage[locale=FR]{siunitx}
\usepackage{tutodoc}



\begin{document}


\section{Tableaux de signes}

\begin{tdocnote}
	Tout comme pour les tableaux de données, c'est via \tdocenv{functab} et un langage spécifique que les tableaux de signes seront fabriqués.
\end{tdocnote}


% --------------- %


\subsection{Le cas des fonctions réelles d'une variable réelle}

La saisie d'un tableau de signes se veut être \tdocquote{naturelle} et \tdocquote{facile}\,. Les exemples suivants devraient vous convaincre que c'est bien le cas.

\end{document}


\end{document}
    