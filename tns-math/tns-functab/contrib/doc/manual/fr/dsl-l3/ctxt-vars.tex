\documentclass[12pt, a4paper]{article}

\usepackage[utf8]{inputenc}
\usepackage[T1]{fontenc}

\usepackage[french]{babel, varioref}

\usepackage{enumitem}
\frenchsetup{StandardItemLabels=true}

\usepackage[lang = french]{tutodoc}



\begin{document}


%\section{La syntaxe du langage proposé}
%\subsection{Les contextes pour les signes et/ou les variations}

\subsubsection{Le contexte \tdocinlatex{vars} pour les variations}

La logique d'utilisation et d'organisation des informations pour le contexte \tdocinlatex{vars} est similaire à celle du contexte \tdocinlatex{signs} ; nous indiquons donc juste les différences.
%
\begin{enumerate}
    \item Voici les informations possibles pour le comportement d'une expression $f$ en supposant que $n$ valeurs pivots \tdocinlatex{x_1} , \tdocinlatex{x_2} , \dots\ , \tdocinlatex{x_n} ont été données via \tdocinlatex{xvals = x_1 , x_2 , ... , x_n} . De nouveau, nous posons $p = n - 1$ .
    %
    \begin{itemize}
        \item A minima, il faut indiquer
        \tdocinlatex{f : v_1 v_2 ... v_p}
        où
        \tdocinlatex{v_1} , \tdocinlatex{v_2} , \dots\ , \tdocinlatex{v_p}
        donnent des informations sur les $p$ intervalles ouverts
        \tdocinlatex{]x_1 ; x_2[} , \tdocinlatex{]x_2 ; x_3[} , \dots\ , \tdocinlatex{]x_p ; x_n[} respectivement.
        Les valeurs possibles pour les \tdocinlatex{v_k} sont les suivantes.
        %
        \begin{enumerate}
            \item \tdocinlatex{<} indique une expression strictement croissante sur l'intervalle concerné.

            \item \tdocinlatex{>} indique une expression strictement décroissante sur l'intervalle concerné.

            \item \tdocinlatex{=} indique une expression constante sur l'intervalle concerné.

            \item \tdocinlatex{u} indique une expression non définie sur l'intervalle concerné (comme pour les signes).
        \end{enumerate}


        \item Pour des valeurs pivots précises, on peut indiquer des images ou des limites à droite et/ou à gauche. Voici ce qui est disponible.
        %
        \begin{enumerate}
            \item L'absence d'expression est possible pour ne rien indiquer du tout.

            \item Toute expression sans lettre \tdocinlatex{u}, ni ponctuation \tdocinlatex{!} est interprétée comme une valeur image au format mathématique \LaTeX.

            \item Une valeur image contenant \tdocinlatex{u} et/ou la ponctuation \tdocinlatex{!} en tant que \tdocquote{token \LaTeX} devra être protégée par des accolades. Ceci vient du fait que dans le langage codant les variations ces deux caractères ont une signification spéciale (voir ci-dessus et l'item suivant).

            \item \tdocinlatex{!} indique que
l'expression n'est pas définie au pivot concerné.
            On peut aussi, si besoin, indiquer des limites à gauche et/ou à droite. Voici les cas possibles.
            %
            \begin{itemize}
                \item \tdocinlatex{!} utilisé seul n'indique aucune limite.


                \item  \tdocinlatex{! d} indique juste $d$ comme limite à droite.

                \textbf{Cette syntaxe est interdite pour le tout dernier pivot.}


                \item \tdocinlatex{g !} indique juste $g$ comme limite à gauche.

                \textbf{Cette syntaxe est interdite pour le tout premier pivot.}


                \item  \tdocinlatex{g ! d} indique $g$ et $d$ comme limites à gauche et à droite respectivement.

                \textbf{Cette syntaxe est interdite pour les premier et dernier pivots.}
            \end{itemize}
        \end{enumerate}
    \end{itemize}
\end{enumerate}


% --------------- %


%Voici un tableau permettant de mémoriser l'organisation des informations de type \tdocquote{variation} et celles de type \tdocquote{pivot}.


% --------------- %


\begin{tdocnote}
    Les espaces autour des doubles points, du point d'exclamation et des informations codées ne sont pas obligatoires.
\end{tdocnote}


% --------------- %


Voici deux codes fictifs illustrant les explications précédentes ; noter au passage que les espaces ignorés permettent d'obtenir un résultat humainement très clair.
\begin{multicols}{2}
    \tdoclatexinput[code]{examples/dsl-l3/vars-always-defined.tex}

    \tdoclatexinput[code]{examples/dsl-l3/vars-with-undefined.tex}
\end{multicols}

Dans le cadre de processus automatisés, il est possible de produire les horreurs suivantes qui aboutiront aux mêmes sorties que les codes correspondants ci-dessus.
\begin{multicols}{2}
    \tdoclatexinput[code]{examples/dsl-l3/vars-always-defined-monster.tex}

    \tdoclatexinput[code]{examples/dsl-l3/vars-with-undefined-monster.tex}
\end{multicols}

\end{document}


\end{document}
    