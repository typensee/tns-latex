\documentclass[12pt, a4paper]{article}

\usepackage[utf8]{inputenc}
\usepackage[T1]{fontenc}

\usepackage[french]{babel, varioref}

\usepackage{enumitem}
\frenchsetup{StandardItemLabels=true}

\usepackage[lang = french]{tutodoc}



\begin{document}


%\section{La syntaxe du langage proposé}

\subsection{Les blocs d'instruction}

L'organisation des blocs d'informations se fait en respectant les règles suivantes.
%
\begin{enumerate}
    \item Les commentaires se font uniquement à la sauce \LaTeX\ via \tdocinlatex{% Mon commentaire} par exemple.


    \item Les lignes vides sont autorisées
    \footnote{
        \label{tns-math-functab-dsl-l3-blank-struct}%
        Ceci est très pratique pour aérer et organiser le code.
    }.


    \item La syntaxe pour les contenus est la suivante.
    %
    \begin{itemize}
        \item Les différents contenus sont séparés par des points-virgules, les retours à la ligne étant autorisés.

        \item Les espaces initiaux et finaux sont ignorés (voir la note \ref{tns-math-functab-dsl-l3-blank-struct} de bas de page).

        \item Tout contenu est associé à un contexte.

        \item Un nouveau contexte s'indique en début de contenu via \tdocinlatex{nom_contexte = contenu_1 ; ...} sans nécessité d'indiquer à chaque fois le contexte.

        \item \textbf{Aucun contexte par défaut n'est initialement défini.}
    \end{itemize}


    \item Le point-virgule est optionnel à la fin du tout dernier contenu.
\end{enumerate}


% --------------- %


\medskip


Voici un code fictif illustrant les règles précédentes.

\tdoclatexinput[code]{examples/dsl-l3/struct-fictional.tex}


Il est possible d'écrire des contenus à la suite comme dans l'exemple suivant
\footnote{
    Nous déconseillons ce style de codage car cela nuit à la lisibilité des instructions.
}.

\tdoclatexinput[code]{examples/dsl-l3/struct-fictional-compact.tex}


On peut même utiliser l'horreur suivante
\footnote{
    Ce mode de rédaction est en fait idéal pour la fabrication automatisée de codes par un script extérieur.
}.

\tdoclatexinput[code]{examples/dsl-l3/struct-fictional-monster.tex}


\end{document}


\end{document}
    