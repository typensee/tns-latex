\documentclass[12pt, a4paper]{article}

\usepackage[utf8]{inputenc}
\usepackage[T1]{fontenc}

\usepackage[french]{babel, varioref}

\usepackage{enumitem}
\frenchsetup{StandardItemLabels=true}

\usepackage[lang = french]{tutodoc}



\begin{document}


%\section{La syntaxe du langage proposé}
%\subsection{Les contextes pour les images de plusieurs fonctions à une variable}

\subsubsection{Le contexte \tdocinlatex{imgs} pour les images}

Le contexte \tdocinlatex{imgs} s'utilise de façon similaire au contexte \tdocinlatex{xvals} mais avec la possibilité de l'employer autant de fois que nécessaire.
Par contre, il est obligatoire d'indiquer l'expression  via
\tdocinlatex{mon_expr : img_1 ,}
\tdocinlatex{img_2 , ... , img_n}
où chaque \tdocinlatex{img_k} peut être soit \tdocquote{vide}\,, soit une expression mathématique \LaTeX\,.
\textbf{Bien retenir qu'il n'existe pas d'expression par défaut.}


% --------------- %


\begin{tdocnote}
    Les espaces autour des doubles points et des virgules ne sont pas obligatoires.
\end{tdocnote}


% --------------- %


Voici un code fictif illustrant les explications précédentes.

\tdoclatexinput[code]{examples/dsl-l3/imgs-f-n-co.tex}


Dans le cadre de processus automatisés, il est possible de produire l'horreur suivante qui aboutira à la même sortie que le code ci-dessus.

\tdoclatexinput[code]{examples/dsl-l3/imgs-f-n-co-monster.tex}


\end{document}


\end{document}
    