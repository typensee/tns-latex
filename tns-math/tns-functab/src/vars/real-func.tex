\documentclass[10pt, a4paper]{article}

\usepackage[utf8]{inputenc}
\usepackage[T1]{fontenc}

\usepackage[french]{babel, varioref}

\usepackage{enumitem}
\frenchsetup{StandardItemLabels=true}

\usepackage[lang = french]{tutodoc}


\usepackage{../main/main}

% TESTING LOCAL IMPLEMENTATION %

\usepackage{vars}


\begin{document}

\section{Tableaux de variations}

\begin{tdocnote}
	Pour les variations, c'est de nouveau via \tdocenv{functab} et un langage spécifique que les tableaux seront fabriqués.
\end{tdocnote}


% --------------- %


\subsection{Le cas des fonctions réelles d'une variable réelle}

La saisie d'un tableau de variations est moins immédiate que celle d'un tableau de signes puisque l'on peut indiquer plus d'informations. Les exemples ci-après ont pour mission de rendre \tdocquote{non douloureuse} la saisie de tableaux de variations à la sauce \thispack.

\end{document}
