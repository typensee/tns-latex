\documentclass[10pt, a4paper]{article}

\newcommand\thispack{\tdocpack{tns-functab}}


\usepackage[utf8]{inputenc}
\usepackage[T1]{fontenc}

\usepackage{multicol}
\usepackage[inline]{enumitem}

\usepackage[french]{babel, varioref}
\frenchsetup{StandardItemLabels=true}

\usepackage[lang = french]{tutodoc}

\usepackage{amsmath}
\usepackage[locale=FR]{siunitx}
\usepackage{tutodoc}


\usepackage{../main/main}

% TESTING LOCAL IMPLEMENTATION %

\usepackage{\jobname}


\begin{document}

\section{Indiquer les valeurs pivots d'un tableau}

Tous les tableaux sont fabriqués via \tdocenv{functab} en utilisant un langage spécifique simplifiant la saisie des informations.
Pour tous les tableaux disponibles, il faut toujours donner des valeurs \tdocquote{pivots}
\footnote{
	 On peut indiquer soit des valeurs dont on veut donner les images, soit des bornes d'intervalles pour les signes et/ou les variations d'une fonction réelle.
} :
cela se fait via 
\tdocinlatex{bounds = ma_var : x_1 , x_2 , ... , x_n}
ou juste 
\tdocinlatex{bounds = x_1 , x_2 , ... , x_n}
si $x$, la variable par défaut, convient
\footnote{
	En fait, \tdocquote{the interval bounds} signifie \tdocinEN{les bornes de l'intervalle}.
	Comme la raison d'être du package \thispack\ est la production simplifiée de tableaux de signes et/ou de variations, le nom \tdocinlatex{bounds} a été retenu au lieu du franco-anglais \tdocinlatex{pivots} par exemple.
}.


% --------------- %


\begin{tdocexa}[Avec la variable par défaut]
    Noter que tout se saisie en mode mathématique.

    \tdoclatexinput{examples/bounds/no-label.tex}
\end{tdocexa}


% --------------- %


\begin{tdocexa}[Avec une variable \tdocquote{maison}]
    \leavevmode

    \tdoclatexinput{examples/bounds/label.tex}
\end{tdocexa}


% --------------- %

\begin{tdocexa}[Nombres décimaux en version \tdocquote{locale} et \tdocquote{grandes} fractions]
%    \leavevmode
%
    L'exemple suivant utilise les macros \tdocmacro{dfrac} et \tdocmacro{num}
    \footnote{
    	Cette macro ajoute de fins espaces mettant en valeur les groupes de chiffres, tout en gérant l'absence d'espaces autour du séparateur décimal.
    }
    venant des excellents packages \tdocpack{amsmath} et \tdocpack{siuntix} respectivement.

    \tdoclatexinput{examples/bounds/dfrac-decimal.tex}
\end{tdocexa}


% --------------- %


\begin{tdocexa}[Commentaires à la sauce \LaTeX]
    \leavevmode

    \tdoclatexinput{examples/bounds/comments.tex}
\end{tdocexa}


% --------------- %


\begin{tdocwarn}
	L'utilisation de \tdocinlatex{bounds} doit se faire obligatoirement une fois, et une seule, au tout début du contenu.
\end{tdocwarn}


% --------------- %


\begin{tdocnote}
	L'environnement \tdocenv{functab} est assez \tdocquote{malin} pour deviner le type de tableau souhaité en fonction des instructions fournies comme nous le constaterons dans les sections plus utiles qui vont suivre.
\end{tdocnote}

\end{document}
