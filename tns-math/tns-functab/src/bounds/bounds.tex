\documentclass[10pt, a4paper]{article}

\newcommand\thispack{\tdocpack{tns-functab}}


\usepackage[utf8]{inputenc}
\usepackage[T1]{fontenc}

\usepackage{multicol}
\usepackage[inline]{enumitem}

\usepackage[french]{babel, varioref}
\frenchsetup{StandardItemLabels=true}

\usepackage[lang = french]{tutodoc}

\usepackage{amsmath}
\usepackage[locale=FR]{siunitx}
\usepackage{tutodoc}


\usepackage{../main/main}

% TESTING LOCAL IMPLEMENTATION %

\usepackage{\jobname}


\begin{document}

\section{Indiquer les valeurs pivots d'un tableau}

Que ce soit pour indiquer les valeurs dont on veut donner les images, ou bien pour donner les bornes des intervalles pour les signes et/ou les variations d'une fonction réelle, il faut donner ces valeurs \tdocquote{pivots}\,. Les exemples suivants montrent que cela se fait via \tdocenv{functab} en utilisant une syntaxe spéciale du type 
\tdocinlatex{bounds = ma_var : x_1 , x_2 , ... , x_n} ,
ou juste 
\tdocinlatex{bounds = x_1 , x_2 ,} \tdocinlatex{... , x_n}
si $x$ la variable par défaut convient
\footnote{
	\tdocquote{the interval bounds} signifie \tdocinEN{le bornes de l'intervalle}.
	Comme la raison d'être du package \thispack\ est la production simplifiée de tableaux de signes et/ou de variations, le nom \tdocinlatex{bounds} a été retenu au lieu du franco-anglais \tdocinlatex{pivots} par exemple.
}.


% --------------- %


\begin{tdocexa}[Avec la variable par défaut]
    Noter que, bien entendu, tout se saisie en mode mathématique.

    \tdoclatexinput{examples/bounds/no-label.tex}
\end{tdocexa}


% --------------- %


\begin{tdocexa}[Avec une variable \tdocquote{maison}]
    \leavevmode

    \tdoclatexinput{examples/bounds/label.tex}
\end{tdocexa}


% --------------- %

\begin{tdocexa}[Les dépendances \tdocpack{mathtools} et \tdocpack{siuntix}]
    L'exemple suivant utilise les macros \tdocmacro{num}
    \footnote{
    	Cette macro ajoute de fins espaces mettant en valeur les groupes de chiffres.
    }
    et \tdocmacro{dfrac} venant des excellents packages \tdocpack{siuntix} et \tdocpack{amsmath} respectivement, le second package étant chargé par l'indispensable \tdocpack{mathtools}.

    \tdoclatexinput{examples/bounds/dep-packs.tex}
\end{tdocexa}


% --------------- %


\begin{tdocexa}[Commentaires à la sauce \LaTeX]
    \leavevmode

    \tdoclatexinput{examples/bounds/comments.tex}
\end{tdocexa}


% --------------- %


\begin{tdocwarn}
	L'utilisation de \tdocinlatex{bounds = ...} doit se faire obligatoirement une fois, et une seule, au tout début du contenu.
\end{tdocwarn}


% --------------- %


\begin{tdocnote}
	L'environnement \tdocenv{functab} est assez \tdocquote{malin} pour deviner le type de tableau souhaité en fonction des instructions fournies comme nous le constaterons dans les sections plus utiles qui vont suivre.
\end{tdocnote}

\end{document}
