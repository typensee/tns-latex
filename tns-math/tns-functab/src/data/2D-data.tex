\documentclass[10pt, a4paper]{article}

\newcommand\thispack{\tdocpack{tns-functab}}


\usepackage[utf8]{inputenc}
\usepackage[T1]{fontenc}

\usepackage{multicol}
\usepackage[inline]{enumitem}

\usepackage[french]{babel, varioref}
\frenchsetup{StandardItemLabels=true}

\usepackage[lang = french]{tutodoc}

\usepackage{amsmath}
\usepackage[locale=FR]{siunitx}
\usepackage{tutodoc}


\usepackage{../main/ctxt-dsl}
\usepackage{../main/main}


% TESTING LOCAL IMPLEMENTATION %

\usepackage{data}


\begin{document}

%\section{Des tableaux de données}

\subsection{Tableaux d'images d'une seule fonction à deux variables}

Voyons comment taper un tableau d'images d'une seule fonction à deux variables, ou de façon équivalente d'un tableau à double-entrée.
\footnote{
	Bien entendu, on perd ici la possibilité de travailler avec plusieurs fonctions au sein du même tableau.
}


% --------------- %


\begin{tdocexa}[Avec des étiquettes de partout]
    En plus de \tdocinlatex{xvals} pour les colonnes, il faut utiliser \tdocinlatex{yvals} pour renseigner les valeurs pivots qui seront utilisées pour chaque ligne.
    De plus, les valeurs de chaque cellule se donnent de façon naturelle sous forme matricielle via \tdocinlatex{mat} : pour une ligne, les valeurs sont séparées par des virgules, tandis que les points-virgules permettent de passer à la ligne suivante.

    \tdoclatexinput{examples/2D-data/label.tex}
\end{tdocexa}


% --------------- %


\begin{tdocexa}[Avec les variables par défaut]
    \leavevmode

    \tdoclatexinput{examples/2D-data/no-label.tex}
\end{tdocexa}


\end{document}
