\documentclass[10pt, a4paper]{article}

\usepackage[utf8]{inputenc}
\usepackage[T1]{fontenc}

\usepackage[french]{babel, varioref}

\usepackage{enumitem}
\frenchsetup{StandardItemLabels=true}

\usepackage[lang = french]{tutodoc}


\usepackage{../main/main}


\begin{document}

%\section{La syntaxe du langage proposé}
%\subsection{Les contextes pour les images d'une seule fonction à deux variables}

\subsubsection{Le contexte \tdocinlatex{mat} pour les images}

Le contexte \tdocinlatex{mat} qui n'est utilisable qu'une seule fois, permet de donner la matrice des données, et ceci en indiquant, si besoin uniquement
\footnote{
	Ceci permet de taper un simple tableau à double entrée.
},
l'expression étudiée.
Voici un code fictif illustrant la logique de saisie.

\tdoclatexinput[code]{examples/dsl-l3/mat.tex}


% --------------- %

Comme les espaces autour des ponctutations ne sont pas obligatoires, dans le cadre de processus automatisés, il sera possible de produire l'horreur suivante qui aboutira à la même sortie que le code ci-dessus.

\tdoclatexinput[code]{examples/dsl-l3/mat-monster.tex}


\end{document}
