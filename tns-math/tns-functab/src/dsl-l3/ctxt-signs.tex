\documentclass[10pt, a4paper]{article}

\newcommand\thispack{\tdocpack{tns-functab}}


\usepackage[utf8]{inputenc}
\usepackage[T1]{fontenc}

\usepackage{multicol}
\usepackage[inline]{enumitem}

\usepackage[french]{babel, varioref}
\frenchsetup{StandardItemLabels=true}

\usepackage[lang = french]{tutodoc}

\usepackage{amsmath}
\usepackage[locale=FR]{siunitx}
\usepackage{tutodoc}


%\usepackage{../macroenv/macroenv}


\begin{document}

%\section{La syntaxe du langage proposé}
%\subsection{Les contextes pour les tableaux de signes et/ou de variations}

\subsubsection{Le contexte \tdocinlatex{signs} pour les signes}

Voici tous les éléments de syntaxe propres au contexte \tdocinlatex{signs}.
%
\begin{enumerate}
    \item Le contexte \tdocinlatex{signs} est employable autant de fois que nécessaire.


	\item L'expression dont on indique le signe doit être donnée via \tdocinlatex{mon_expr : ...} où les points de suspension spécifient les informations liées au comportement de l'expression (voir l'item suivant à ce sujet).
	\textbf{Il n'existe pas d'expression par défaut.}


    \item Dans cet item, pour expliquer ce qui est attendu comme informations relatives au comportement d'une expression $f$ , nous allons supposer que $n$ valeurs pivots \tdocinlatex{x_1} , \tdocinlatex{x_2} , \dots\ , \tdocinlatex{x_n} ont été données via \tdocinlatex{bounds = x_1 , x_2 , ... , x_n} . Nous posons aussi $p = n - 1$ avec \tdocpre{p} pour \tdocquote{\tdocprewhy{p.récédent} le naturel $n$} .
    %
    \begin{itemize}
        \item A minima, il faut indiquer 
        \tdocinlatex{f : s_1 s_2 ... s_p} 
        où \tdocinlatex{s_1} , \tdocinlatex{s_2} , \dots\ , \tdocinlatex{s_p} 
        donnent des informations sur les $p$ intervalles 
        \tdocinlatex{]x_1 ; x_2[} , \tdocinlatex{]x_2 ; x_3[} , \dots\ , \tdocinlatex{]x_p ; x_n[} respectivement.
        Les valeurs possibles pour les \tdocinlatex{s_k} sont les suivantes.
		%
		\begin{enumerate}
			\item \tdocinlatex{+} indique une expression positive stricte sur l'intervalle concerné.
			
			\item \tdocinlatex{-} indique une expression négative stricte sur l'intervalle concerné.
			
			\item \tdocinlatex{u} indique une expression non définie sur l'intervalle concerné avec \tdocpre{u} pour \tdocprewhy{u.ndefined} soit \tdocinEN{non défini}
.
	
			\item \tdocinlatex{z} indique une expression nulle sur l'intervalle concerné avec \tdocpre{z} pour \tdocprewhy{z.éro}.
		\end{enumerate}

        
        \item On peut aussi indiquer le comportement d'une expression en certaines valeurs pivots. Ceci se fait à côté d'une information de type signe : par exemple, en gardant les notations de l'item précédent, nous avons les possibilités suivantes.
        %
        \begin{itemize}
        	\item Si $n > 3$ alors $f(x_3) = 0$ s'indique via \tdocinlatex{... 0 s_3 ...}
	
        	\item Si $n = 3$ alors $f(x_3) = 0$ s'indique via \tdocinlatex{... s_2 0}
	
        	\item $f(x_1) = 0$ s'indique via \tdocinlatex{0 s_1 ...}	
        \end{itemize}
        %
        Les valeurs possibles pour le comportement éventuel en une valeur pivot sont les suivantes.
		%
		\begin{enumerate}
			\item \tdocinlatex{0} indique que 
l'expression s'annule au pivot concerné.
			
			\item \tdocinlatex{!} indique que 
l'expression n'est pas définie au pivot concerné
        	\footnote{
        		En France, le panneau de signalisation indiquant un danger est un triangle blanc au bords rouges rouge contenant un point d'exclamation.
			}.
		\end{enumerate}
    \end{itemize}
\end{enumerate}


% --------------- %


\begin{tdocnote}
    Les espaces autour des doubles points et des informations codées ne sont pas obligatoires.
\end{tdocnote}


% --------------- %


Voici deux codes fictifs illustrant les indications précédentes ; noter au passage que les espaces ignorés permettent d'obtenir un résultat humainement très clair.
\begin{multicols}{2}
	\tdoclatexinput[code]{examples/dsl/signs-plus-minus-zero.tex}

	\tdoclatexinput[code]{examples/dsl/signs-plus-minus-strange.tex}
\end{multicols}

Dans le cadre de processus automatisés, il est possible de produire les horreurs suivantes qui produiront les mêmes sorties que les codes ci-dessus.
\begin{multicols}{2}
	\tdoclatexinput[code]{examples/dsl/signs-plus-minus-zero-monster.tex}

	\tdoclatexinput[code]{examples/dsl/signs-plus-minus-strange-monster.tex}
\end{multicols}

\end{document}
