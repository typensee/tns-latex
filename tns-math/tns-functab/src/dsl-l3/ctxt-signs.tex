\documentclass[10pt, a4paper]{article}

\usepackage[utf8]{inputenc}
\usepackage[T1]{fontenc}

\usepackage[french]{babel, varioref}

\usepackage{enumitem}
\frenchsetup{StandardItemLabels=true}

\usepackage[lang = french]{tutodoc}


%\usepackage{../macroenv/macroenv}


\begin{document}

%\section{La syntaxe du langage proposé}
%\subsection{Les contextes pour les tableaux de signes et/ou de variations}

\subsubsection{Le contexte \tdocinlatex{signs}}

Voici tous les éléments de syntaxe propres au contexte \tdocinlatex{signs}.
\begin{enumerate}
    \item Le contexte \tdocinlatex{signs} est employable autant de fois que nécessaire.


	\item L'expression dont on indique le signe doit être donnée via \tdocinlatex{mon_expr : ...} où les points de suspension spécifient les informations liées au signe de l'expression (voir l'item suivant). 


    \item Dans cet item, nous allons supposer que \tdocinlatex{bounds = x_1 , x_2 , x_3} a été utilisé pour donner les valeurs pivots \tdocinlatex{x_1} , \tdocinlatex{x_2} et \tdocinlatex{x_3} . Les cas suivants couvrent tout ce qu'il est possible de faire.
    \begin{itemize}
        \item \tdocinlatex{f : + -} indique que
        \verb#f > 0# sur \verb#]x_1 ; x_2[# ,
        \verb#f < 0# sur \verb#]x_2 ; x_3[# ,
        sans donner d'information particulière sur
        \verb#f# en \tdocinlatex{x_1} , \tdocinlatex{x_2} et \tdocinlatex{x_3} 
        \footnote{
        	Notons que ceci est totalement étrange.
		}.


        \item \tdocinlatex{f : + 0 -} complète ce qui précède en indiquant que \verb#f = 0# en \verb#x_2# .


        \item \tdocinlatex{f : + ! -} permet d'indiquer une valeur interdite en \verb#x_2#
        \footnote{
        	Le panneau français indiquant un danger est un triangle rouge contenant un point d'exclamation.
		}.


        \item \tdocinlatex{f : 0 + ! - !} est autorisé pour indiquer aussi le comportement de \verb#f# en \verb#x_1# et \verb#x_3# .


        \item \tdocinlatex{f : n z}  indique de ne rien mettre pour la zone relative à \verb#]x_1 ; x_2[# et aussi que \verb#f = 0# sur \verb#]x_2 ; x_3[#
        \footnote{
        	Les lettres \tdocpre{n} et \tdocpre{z} sont pour
        	\tdocprewhy{n.othing} et \tdocprewhy{z.ero}
        	où
        	\tdocquote{nothing} signifie \tdocinEN{rien}.
		}.
    \end{itemize}
\end{enumerate}


% --------------- %


\begin{tdocnote}
    Les espaces autour des doubles points et des virgules ne sont pas obligatoires.
\end{tdocnote}


% --------------- %


Voici deux codes fictifs reprenant les indications précédentes. Noter que les espaces ignorés permettent d'obtenir un résultat humainement clair.
\begin{multicols}{2}	
	\tdoclatexinput[code]{examples/dsl/signs-plus-minus-zero.tex}
	
	\tdoclatexinput[code]{examples/dsl/signs-plus-minus-strange.tex}
\end{multicols}

\end{document}
