\documentclass[10pt, a4paper]{article}

\usepackage[utf8]{inputenc}
\usepackage[T1]{fontenc}

\usepackage[french]{babel, varioref}

\usepackage{enumitem}
\frenchsetup{StandardItemLabels=true}

\usepackage[lang = french]{tutodoc}


\usepackage{../main/main}


\begin{document}

%\section{La syntaxe du langage proposé}

\subsection{Les contextes pour les tableaux d'images}

Les contextes utilisables dans un même tableau de type \tdocquote{images} sont les suivants.

\medskip

\begin{itemize*}[before = \leavevmode\kern15pt, itemjoin = \kern15pt]
	\item \tdocinlatex{bounds}

	\item \tdocinlatex{imgs}
\end{itemize*}


% --------------- %


\subsubsection{Le contexte \tdocinlatex{bounds} pour les valeurs pivots}
\label{tns-math-functab-dsl-l3-ctxt-bounds}

Pour indiquer les valeurs pivots nécessaires à la conception d'un tableau, il faut passer via le contexte \tdocinlatex{bounds} en respectant les règles ci-après.
%
\begin{enumerate}
    \item Le contexte \tdocinlatex{bounds} doit être employé une seule fois et avant tout autre contexte.


    \item Un seul contenu est utilisable en utilisant l'un des deux formats suivants où le nombre de valeurs n'est pas majoré.
    %
    \begin{itemize}
        \item \tdocinlatex{ma_var : x_1 , x_2 , ... , x_n} indique que $n$ valeurs pivots \tdocinlatex{x_1} , \tdocinlatex{x_2} , \dots\ , \tdocinlatex{x_n} ont été données pour la variable nommée \tdocinlatex{ma_var} avec $n \geq 2$ forcément.

        \item \tdocinlatex{x_1 , x_2 , ... , x_n} indique que $n$ valeurs pivots \tdocinlatex{x_1} , \tdocinlatex{x_2} , \dots\ , \tdocinlatex{x_n} ont été données pour la variable nommée \tdocinlatex{x} qui est le nom par défaut.
    \end{itemize}
\end{enumerate}


\medskip


Voici deux codes fictifs illustrant les règles précédentes.
\begin{multicols}{2}
    \tdoclatexinput[code]{examples/dsl-l3/bounds-label.tex}

    \tdoclatexinput[code]{examples/dsl-l3/bounds-no-label.tex}
\end{multicols}


% --------------- %


\begin{tdocnote}
    Les espaces autour des doubles points et des virgules ne sont pas obligatoires bien que conseillés pour des raisons de lisibilité.
\end{tdocnote}

\end{document}





% --------------- %


\begin{tdocwarn}
    La suite de packages \tdocpack{tns} s'appuie sur les conventions mathématiques anglo-saxonnes avec le point comme séparateur décimal. Le plus simple pour travailler avec une virgule comme séparateur décimal et de passer via \tdocinlatex{\num{2.3}} du package qui est automatiquement chargé
\end{tdocwarn}
