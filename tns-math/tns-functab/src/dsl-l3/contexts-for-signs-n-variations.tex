\documentclass[10pt, a4paper]{article}

\usepackage[utf8]{inputenc}
\usepackage[T1]{fontenc}

\usepackage[french]{babel, varioref}

\usepackage{enumitem}
\frenchsetup{StandardItemLabels=true}

\usepackage[lang = french]{tutodoc}


\usepackage{../main/main}


\begin{document}

%\section{La syntaxe du langage proposé}

\subsection{Les contextes pour les tableaux de signes et/ou de variations}

Les contextes utilisables dans un même tableau de type \tdocquote{signes et/ou variations} sont les suivants.

\medskip

\begin{itemize*}[before = \leavevmode\kern15pt, itemjoin = \kern15pt]
	\item \tdocinlatex{xvals}

	\item \tdocinlatex{signs}

	\item \tdocinlatex{vars}
\end{itemize*}


% --------------- %


\subsubsection{Le contexte \tdocinlatex{xvals} pour les valeurs pivots}

Se reporter à la section \ref{tns-math-functab-dsl-l3-ctxt-xvals} donnant la syntaxe attendu ; ce qui échange ici est que les valeurs données sont des valeurs pivots liées aux intervalles sur lesquels on donne des informations.

\end{document}
