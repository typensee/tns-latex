\documentclass[10pt, a4paper]{article}

\usepackage[utf8]{inputenc}
\usepackage[T1]{fontenc}

\usepackage[french]{babel, varioref}

\usepackage{enumitem}
\frenchsetup{StandardItemLabels=true}

\usepackage[lang = french]{tutodoc}


\usepackage{../main/main}


\begin{document}

%\section{La syntaxe du langage proposé}
%\subsection{Les contextes pour les images de plusieurs fonctions à une variable}

\subsubsection{Le contexte \tdocinlatex{xvals} pour les valeurs étudiées}
\label{tns-math-functab-dsl-l3-ctxt-xvals}

Pour indiquer les valeurs nécessaires à la conception d'un tableau, il faut passer via le contexte \tdocinlatex{xvals} en respectant les règles ci-après.
%
\begin{enumerate}
    \item Le contexte \tdocinlatex{xvals} doit être employé une seule fois en tout début de code, hors commentaires, c'est-à-dire avant le contexte \tdocinlatex{imgs}.


    \item Un seul contenu est utilisable en utilisant l'un des deux formats suivants où le nombre de valeurs n'est pas majoré.
    %
    \begin{itemize}
        \item \tdocinlatex{ma_{var} : x_1 , x_2 , ... , x_n} indique que les $n$ valeurs \tdocinlatex{x_1} , \tdocinlatex{x_2} , \dots\ , \tdocinlatex{x_n} ont été données pour la variable nommée $ma_{var}$ avec $n \geq 2$ forcément.

        \item \tdocinlatex{x_1 , x_2 , ... , x_n} indique que $n$ valeurs pivots \tdocinlatex{x_1} , \tdocinlatex{x_2} , \dots\ , \tdocinlatex{x_n} ont été données pour la variable nommée $x$ qui est le nom par défaut.
    \end{itemize}
\end{enumerate}


\medskip


Voici deux codes fictifs illustrant les règles précédentes.
\begin{multicols}{2}
    \tdoclatexinput[code]{examples/dsl-l3/xvals-label.tex}

    \tdoclatexinput[code]{examples/dsl-l3/xvals-no-label.tex}
\end{multicols}


% --------------- %


\begin{tdocnote}
    Les espaces autour des doubles points et des virgules ne sont pas obligatoires bien que conseillés pour des raisons de lisibilité.
\end{tdocnote}

\end{document}
