\documentclass[10pt, a4paper]{article}

\usepackage[utf8]{inputenc}
\usepackage[T1]{fontenc}

\usepackage[french]{babel, varioref}

\usepackage{enumitem}
\frenchsetup{StandardItemLabels=true}

\usepackage[lang = french]{tutodoc}


%\usepackage{../macroenv/macroenv}


\begin{document}

\noindent
Le paquet \thispack\ fournit des moyens intuitifs et efficaces pour taper des tableaux décrivant les signes et les variations de fonctions mathématiques, ou bien de petits tableaux de données.
%remplies éventuellement via des \tdocquote{fonctions informatiques}
%\footnote{
%	La référence à \tdocquote{fonction} est donc polysémique.
%}.
Voici ce qui est proposé pour cette toute première version.
\begin{enumerate}
	\item Tableaux de données remplis à la main.% ou via une macro.

%	\item Tableaux de données pour des suites récursives remplis à la main et/ou via une macro.
%
%	\item Tableaux de signes et/ou de variations de fonctions réelles.
%
%	\item Tableaux de signes et/ou de variations associées à des courbes planes paramétrées réelles.
\end{enumerate}

\medskip

\noindent
Deux points importants à noter.
\begin{itemize}
    \item Sans le paquet \tdocpack{nicematrix} qui fait tout le travail ingrat, le paquet \thispack\ n'aurait certainement pas vu le jour.

    \item Cette documentation est aussi disponible en anglais.
\end{itemize}


% ------------------ %


\tdocsep

{\noindent
\small\itshape
\textbf{Abstract.}
The \thispack\ package provides intuitive and efficient ways of typing tables describing the signs and variations of mathematical functions, or small tables of data. % possibly filled in via \tdocquote{computer functions}
%\footnote{
%	The reference to \tdocquote{function} is therefore polysemous.
%}.
Here is what is proposed for this very first version.
%Here is what is currently proposed.
\begin{enumerate}
	\item Data tables filled in by hand.% or via a macro.

%	\item Tables of data for recursive sequences filled in by hand and/or via a macro.
%
%	\item Tables of signs and/or variations of real functions.
%
%	\item Tables of signs and/or variations associated with real parametric plane curves.
\end{enumerate}

\medskip

\noindent
Two important points to note.
\begin{itemize}
    \item Without the \tdocpack{nicematrix} package which does all the thankless work, the \thispack\ package would certainly not have seen the light of day.

    \item This documentation is also available in French.
\end{itemize}
}

\end{document}
