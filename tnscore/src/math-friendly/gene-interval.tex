\documentclass[12pt, a4paper]{article}

% --- LOCAL IMPORT(S) - START --- %
\makeatletter
    % == PACKAGES USED == %

\RequirePackage{relsize}


% == DEFINITIONS == %

% Math tools - Intervals and co

\newcommand\tns@extra@vphantom{%
    \vphantom{\relsize{1.25}{\text{$\displaystyle F_1^2$}}}%
}


% #1 : left symbol
% #2 : 1st part
% #3 : 2nd part
% #4 : 3rd part
% #5 : right symbol
\newcommand\tns@generic@interval@semi@ext[5]{%
    \ensuremath{%
        \left#1 \tns@extra@vphantom \right. \kern-.25em%
        #2 #3 #4%
        \left. \tns@extra@vphantom \kern-.05em \right#5%
    }%
}


% #1 : left symbol
% #2 : 1st part
% #3 : 2nd part
% #4 : 3rd part
% #5 : right symbol
\newcommand\tns@generic@interval@ext[5]{%
    \ensuremath{%
        \left#1 #2 #3 #4 \right#5%
    }%
}


\makeatother
% --- LOCAL IMPORT(S) - END --- %

\usepackage{amsmath}


\begin{document}


% ---------------------- %


\subsection{Intervalles généralisés}

\makeatletter
\newcommand\strangeset[2]{%
    \tns@generic@interval@ext{|}  % 1e délimiteur
                             {#1} % 1e élément
                             {:.:} % Séparateur entre les deux éléments 
                             {#2} % 2e élément
                             {<}  % 2e délimiteur
}
\makeatother

Il est possible définir facilement des intervalles généralisés constitués de deux délimiteurs extensibles contenant deux expressions choisies par l'utilisateur et séparées par une ponctuation, ou un délimiteur extensible.
Par exemple, il est assez facile d'obtenir ce qui suit, tout en choisissant le comportement des délimiteurs.
\[ \strangeset{a}{b} \text{ ou } \strangeset{a}{\dfrac{b}{c}} \]



%
%La macro \macro{strangeset} a été définie comme suit via \macro{tns@generic@interval@ext}.
%
%\begin{latexex-alone}
%\newcommand\strangeset[2]{%
%    \tns@generic@interval@ext{\{} % 1e délimiteur
%                             {#1} % 1e élément
%                             {::} % Séparateur entre les deux éléments 
%                             {#2} % 2e élément
%                             {)}  % 2e délimiteur
%}
%\end{latexex-alone}
%
%
%% ---------------------- %
%
%
%\newparaexample{Mode semi-extensible}
%
%Le mode semi-extensible correspond à des délimiteurs un peu plus grand qu'en mode non extensible comme le montre l'exemple ci-après.
%
%\makeatletter
%\newcommand\myinter[2]{%
%    \tns@generic@interval@semi@ext{\{}%
%                                  {#1}{::}{#2}%
%                                  {)}%
%}
%\makeatother
%
%\begin{latexex}
%$\myinter{a}{\dfrac{b}{c}}$ ou
%$\{ a :: \dfrac{b}{c} )$
%\end{latexex}
%
%
%Il suffit d'utiliser \macro{tns@generic@interval@semi@ext} au lieu de \macro{tns@generic@interval@ext}. Voici le code utilisé.
%
%\begin{latexex-alone}
%\newcommand\myinter[2]{%
%    \tns@generic@interval@semi@ext{\{}%
%                                  {#1}{::}{#2}%
%                                  {)}%
%}
%\end{latexex-alone}
%
%
%% ---------------------- %
%
%
%\subsection{Fiches techniques}
%
%\IDmacro[a]{tns@generic@interval@ext     }{5}
%
%\IDmacro[a]{tns@generic@interval@semi@ext}{5}
%
%
%\IDarg{1} le 1\ier{} délimiteur qui est à gauche.
%
%\IDarg{2} le 1\ier{} élément de l'intervalle généralisé.
%
%\IDarg{3} le séparateur entre les deux éléments.
%
%\IDarg{4} le 2\ieme{} élément de l'intervalle généralisé.
%
%\IDarg{5} le 2\ier{} délimiteur qui est à droite.

\end{document}