\section{Cas d'utilisation tapés directement}

\subsection{Face à face}

\begin{doclatex-alone}
\begin{doclatex}
    $A = B + C$
\end{doclatex}
\end{doclatex-alone} 

Ceci donne :

\begin{doclatex}
    $A = B + C$
\end{doclatex}


% ------------------ %


\subsection{À la suite}

\begin{doclatex-alone}
\begin{doclatex-flat}
    $A = B + C$
\end{doclatex-flat}
\end{doclatex-alone}

Ceci donne :

\begin{doclatex-flat}
    $A = B + C$
\end{doclatex-flat}


% ------------------ %


\subsection{Juste le code}

En utilisant \verb+doclatex-alone+ au lieu de \verb+doclatex-flat+, on obtient :

\begin{doclatex-alone}
    $A = B + C$
\end{doclatex-alone}


% ------------------ %


\section{Cas d'utilisation importés}

Pour les codes suivants, on part du principe que le fichier \verb+showcase/examples/xyz.tex+ relativement au document présent contient juste la ligne \verb+$x y z = 1$+ .

\subsection{Face à face}

\begin{doclatex-alone}
\inputdoclatex{showcase/examples/xyz.tex}
\end{doclatex-alone}

Ceci donne :

\inputdoclatex{showcase/examples/xyz.tex}


% ------------------ %


\subsection{À la suite}

\begin{doclatex-alone}
\inputdoclatexflat{showcase/examples/xyz.tex}
\end{doclatex-alone}

Ceci donne :

\inputdoclatexflat{showcase/examples/xyz.tex}


% ------------------ %


\subsection{Juste le code}

\begin{doclatex-alone}
\inputdoclatexalone{showcase/examples/xyz.tex}
\end{doclatex-alone}

Ceci donne :

\inputdoclatexalone{showcase/examples/xyz.tex}


% ------------------ %


\subsection{Code suivi du rendu centré}

\begin{doclatex-alone}
\inputdoclatexbefore{showcase/examples/xyz.tex}
\end{doclatex-alone}

Ceci donne :

\begin{docshowcase}
    \inputdoclatexbefore{showcase/examples/xyz.tex}
\end{docshowcase}


% ------------------ %


\subsection{Rendu centré suivi du code}\label{output-centered}

\begin{doclatex-alone}
\inputdoclatexafter{showcase/examples/xyz.tex}
\end{doclatex-alone}

Ceci donne :

\begin{docshowcase}
    \inputdoclatexafter{showcase/examples/xyz.tex}
\end{docshowcase}


% ------------------ %


\subsection{Rendu réel}

Parfois on veut pouvoir montrer du code en situation réelle comme dans les deux exemples précédents. Voici comment faire.


\docexa{Juste le rendu}

Le rendu de l'exemple de la section précédente \ref{output-centered} a été tapé comme suit.

\begin{doclatex-alone}
\begin{docshowcase}
    \inputdoclatexafter{showcase/examples/xyz.tex}
\end{docshowcase}
\end{doclatex-alone}


% ------------------ %


\docexa{Code importé et son rendu}

\begin{doclatex-alone}
    \inputdoclatexreal{showcase/examples/xyz.tex}
\end{doclatex-alone}

Ceci va produire :

\begin{showcase}
	\inputdoclatexreal{showcase/examples/xyz.tex}
\end{showcase}

