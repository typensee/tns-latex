\documentclass[12pt,a4paper]{article}

\usepackage[utf8]{inputenc} 

\usepackage[french]{babel, varioref}

% USING OTHER TOOLS % 
\usepackage[lang = FR]{../main/main}
\usepackage{../macroenv/macroenv}
\usepackage{../listing/listing}
\usepackage{../rem-exa/rem-exa}


% TESTING LOCAL IMPLEMENTATION % 
\usepackage{inenglish}


\begin{document}

\section{Cela veut dire quoi en \docquote{angliche}}

Penser aux non-anglophones est bien même si ces derniers se font de plus en plus rares.

\begin{doclatex-flat}
Cool et top signifient \docinEN*{cool} et \docinEN{top}.
\end{doclatex-flat}


La macro \docmacro{docinEN} et sa version étoilée s'appuient sur \docmacro{docquote} : par exemple, le code \docilatex|\docquote{sémantique}| produit \docquote{sémantique}.


\begin{docrem}
	Les guillemets sont accessibles via les macros \docilatex|\og| et \docilatex|\fg| .
\end{docrem}


\end{document}

